\section{Newton Vs. Einstein}

\begin{frame}{Newton Vs. Einstein}{Newton's Laws}
	According to newton's laws we know that
	$$ F = m_i a $$
	where $m_i$ is inertial mass.

	Newton also proposed the Gravitational Law:
	$$ F =- \frac{GM_g m_g}{r^2} $$

	Equivalence law states that the property that determines, how strongly
	an object is pulled by gravity, also tells us how it accelerates by ANY
	force. Proof is simple observations made by Galileo on objects on earth.

	Using Guass's law and other vector algebra we can define a quantity that
	measures how much Gravity acts on a given point in space using
	\textbf{Poisson's Law}:
	$$ \nabla^2 \Phi = 4\pi G \rho $$
\end{frame}

\begin{frame}
	This quantity is called the Gravitational potential $\Phi$, and is
	divergence of the acceleration due to gravity of a test mass,
	$$ \vec a = - \vec \nabla \cdot \vec \Phi $$

	This Poission's equation basically relates Gravity with mass density.
\end{frame}


\begin{frame}{Newton Vs. Einstein}{Special Theory of Relativity}
	Postulates:
	\begin{itemize}[<+->]
		\item In all intertial frames (where laws of physics are valid),
			the basic laws of physics are same.
		\item If above is true, then for all the frames Maxwell's laws
			are true and we calculate the speed of light to be c.
        \end{itemize}

	This means say in two different frames, the displacement of light in both
	of these frames must be equal.
	$$ c^2t^2 = x^2 + y^2 + z^2 $$
	$$ c^2t'^2 = x'^2 + y'^2 + z'^2 $$

	Thus we have to use the lorentz tranformation instead:
	$$ x' = \gamma(x - vt), t' = \gamma(t - vx/c^2), y'=y, z'=z $$
	Where $\gamma = (1 - v^2/c^2)^{-1/2}$

\end{frame}

\begin{frame}
	So here, we need to find a quantity that is constant accross these
	intertial frames. This quantity is called spacetime (minkowski metric)

	$$ ds^2 = ct^2 - dx^2 - dy^2 - dz^2 $$
	or simply
	$$ ds^2 = ct^2 - dl^2 $$

	Light is said to travel a null geodesic path, which is in accordance
	with the fermat's principle that, light always takes the path with
	shortest time (now spacetime). Thus $ds^2 = 0$
\end{frame}


\begin{frame}{Newton Vs. Einstein}{General Theory of Relativity}
	The above calculations are considering that there is no curvature in
	space and there is no gravity acting.

	Einstein used the
	Equivalence principle to explain that light must bend due to gravity
	even though it has no mass.

	Thus Einstein concludes that it is not mass that makes things bend but
	the Energy. Mass and energy for non-relativistic objects are
	interconvertible using the $E = mc^2$, and Energy is interconvertible
	with momentum for relativistic objects as $E = pc$
\end{frame}


\begin{frame}{Newton Vs. Einstein}{Defining Curvature}
	There are two types of uniform curvatures, postive and negetive (taking
	2-D for example):
	\begin{itemize}
		\item  Uniform Positive Curvature:
			$$ \alpha + \beta + \gamma = \pi + A/R_0^2  $$
			When we measure a distance $dl$ along this curved
			surface, then interms of spherical coordinates it can be
			written as:
			$$ dl^2 = dr^2 + R^2 sin^2(r/R)d\theta^2 $$
		\item  Uniform Negetive Curvature:
			$$ \alpha + \beta + \gamma = \pi - A/R_0^2  $$
			When we measure in spherical coordinates, the value
			comes out to be
			$$ dl^2 = dr^2 + R^2 sinh^2(r/R)d\theta^2 $$

        \end{itemize}
\end{frame}

\begin{frame}{Newton Vs. Einstein}{Defining Curvature}
	When we try to scale this to 3-D we get:

	Flat:
	Uniform positive curvature gives:
	$$ dl^2 = dr^2 + r  [d\theta^2 + sin^2 \theta d\phi^2]$$
	Uniform positive curvature gives:
	$$ dl^2 = dr^2 + R^2 sin^2(r/R)  [d\theta^2 + sin^2 \theta d\phi^2]$$
	Negetive positive curvature gives:
	$$ dl^2 = dr^2 + R^2 sinh^2(r/R)  [d\theta^2 + sin^2 \theta d\phi^2]$$

	We can generalize this by substituiting variables like $S_k(r)$ and
	$d\Omega$. Where
	$$ d\Omega = d\theta^2 + sin^2\theta d\phi^2 $$
\end{frame}

\begin{frame}
	$$ S_k(r) = \begin{cases}
		Rsin(r/R) & (\kappa = 1) \\
		r & (\kappa = 0) \\
		Rsinh(r/R) & (\kappa = -1) \\
	\end{cases}$$
	Which makes the distance equation as:
	$$ dl^2 = dr^2 + S_k(r)^2 d\Omega^2 $$

\end{frame}

\begin{frame}{Newton Vs. Einstein}{Robertson-Walked Metric}
	This metric is a curved generalization to the Minkowski metric given by
	$$ ds^2 = -c^2 dt^2 + a(t)^2 [ dr^2 + S_k(r)^2 d\Omega^2 ] $$

	This is important because it takes into account that, the expansion
	coefficient determines how the universe moves.
\end{frame}


\begin{frame}{Newton Vs. Einstein}{Proper Distance}
	When we measure distance in a curved / expanding universe we don't take
	into account the error which is caused because of the expansion
	coefficient. Proper distance does.
	$$ ds^2  = a(t)^2 [dr^2 + S_\kappa (r)^2 d\Omega^2] $$
	Taking angular component to be zero if we observe object directly.
	$$ ds = a(t)dr $$
	$$ d_p(t) = a(t)r $$

	If you observe this proves hubble's law:
	$$ v_p(t_0) = H_0 d_p(t_0) $$
\end{frame}

\begin{frame}{Newton Vs. Einstein}{Expansion and red Shift}

	We can use the fact that photons follow geodesic. In the
	robertson-walker equation we have:
	$$ 0 = -cdt^2 + a(t)^2 (dr)^2 $$
	$$ a(t).dr = c.dt $$
	$$ \int^{t_o}_{t_e}\frac{dt}{a(t)} = \int^{r}_{0} dr = r $$
	Integrating between time of observation and emission.

	If we were to calculate for the next crest of the wave we will still get
	r.
	Thus by subtracting we get.
	$$ \int_{t_e}^{t_o} \frac{dt}{a(t)} = \int_{t_e + \lambda_e / c}^{ t_o + \lambda_o / c } \frac{dt}{a(t)} $$
	$$ \int^{t_e + \lambda_e / c}_{t_e} \frac{dt}{a(t)} = \int^{ t_o +
	\lambda_o / c }_{t_o} \frac{dt}{a(t)} $$
\end{frame}

\begin{frame}
	Using the fact that expansion of universe barely happens when the crest
	propogates we use:
	$$ \frac{\lambda_e}{a(t_e)} =  \frac{\lambda_o}{a(t_o)}$$
	$$ 1 + z =  \frac{1}{a(t_e)} $$
\end{frame}
