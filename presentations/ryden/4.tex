\section{Cosmic Dynamics}

\begin{frame}{Cosmic Dynamic}{Einstien's Field Equation}
	$$ G_{\mu \nu} = \frac{8\pi G}{c^4} T_{\mu \nu} $$
	The $G$ term is called Einstein's metric tensor (describes the curvature
	of universe at all x,y,z,t) and the T term is called
	the stress-energy metric tensor.
\end{frame}

\begin{frame}{Cosmic Dynamic}{Friedmann's equation}
	An equation that links the Expansion of universe, Curvature and the
	Mass-energy was required including terms $a(t), R_0, \kappa,
	\epsilon(t)$.


	$$ \left( \frac{\dot a}{a} \right)^2 = \frac{8\pi G}{3c^2}\epsilon(t) -
	\frac{\kappa c^2}{R_0^2} \frac{1}{a(t)^2}$$

	Friedmann proved this for General theory of relativity, but we will
	prove it using the Newtonian Way.

	To get to this, start with this take, a sphere that is contracting
	because of its own action of gravity. Acceleration can be given as
	$$ \frac{d^2R_s(t)}{dt^2} = - \frac{GM_s}{R(t)^2} $$


\end{frame}

\begin{frame}{Cosmic Dynamic}{Critical Density}
	We can calculate the energy density using above relation for a good
	estimate case senario where $\kappa = 0$, and $H(t) = \frac{\dot a}{a}$
	is the hubble parameter. We get

	$$ \epsilon_c(t) = \frac{3c^2}{8\pi G} H(t)^2  $$
	For the current time, we can substituite the Hubble's constant and get
	$\epsilon_{c,0} = 4870 MeV/m^3$ which also gives us mass density if you
	divide by $c^2$.

	We define Density parameter to be that of ratio with the critical
	density for any universe as:
	$$ \Omega (t)  = \frac{\epsilon(t)}{\epsilon_c(t)} = 1 + \frac{\kappa
	c^2}{R_0^2 a(t)^2 H(t)^2} $$

	Using we get the relation to finally get rid of $\kappa$:
	$$ \frac{\kappa}{R_0^2} = \frac{H_0^2}{c^2} (\Omega_0 - 1) $$
\end{frame}


\begin{frame}{Cosmic Dynamic}{Fluid Equation and Acceleration Equation}
	We know that Friedmann equation in the end of the day comes from energy
	conservation, so we can use the first law of  thermodynamics to get a
	similar equation.

	We start with
	$$ \dot E + P \dot V = 0 $$
	and finally get:
	$$ \dot \epsilon + 3 \frac{\dot a}{a} (\epsilon + P) = 0 $$

	We can use this in the friedmann equation to get the acceleration
	equation.
	$$ \frac{\ddot a}{a} = \frac{4\pi G}{3c^2}\left( \epsilon + 3P \right) $$
\end{frame}

\begin{frame}{Cosmic Dynamic}{Equations of State}
	We need another set of equations to solve the friedmann equation:
	This equation called called equation of state is a relation between the
	pressure used in fluid equation and energy density
	$$ P = w \epsilon  $$

	This w can vary with the substance used. For $w = 0$ for matter (can be
	solved by using ideal gas equation) and for radiation the $w = 1/3$
	comes from statistical mechanics and boson nature.
\end{frame}


\begin{frame}{Cosmic Dynamic}{Cosmological constant}

	Einstein first introduced this to explain the static nature of universe
	but cosmologists now use it because the default calculation of hubble's
	constant are very underestimated. When added to the friedmann equation,
	the cosmological constant looks like
	$$ \left( \frac{\dot a}{a} \right)^2 = \frac{8\pi G}{3c^2}\epsilon(t) -
	\frac{\kappa c^2}{R_0^2} \frac{1}{a(t)^2} + \frac{\Lambda}{3}$$

	Its as if cosmological constant had its own energy density which doesn't
	very with time or expansion.
	$$ \left( \frac{\dot a}{a} \right)^2 = \frac{8\pi G}{3c^2}(\epsilon(t) +
	\epsilon_\Lambda )-
	\frac{\kappa c^2}{R_0^2} \frac{1}{a(t)^2}$$

	If we were to call, $\epsilon_k = \epsilon + \epsilon_\Lambda$, that
	would mean to keep the fluid equation constant we need. $w = -1$, which
	means there is an attrative force.



\end{frame}

\begin{frame}

	This is same as what Einstein did to add $\Lambda$ to the Newton's
	equation. (acceleration equation, when einstein's universe is static)
	$$ 0 = - \frac{4\pi G}{3} + \frac{\Lambda}{3} $$
	When we solve for friedmann equation too (with static universe) we get.
	$$ R_0 = \frac{c}{\Lambda^{1/2}} $$
\end{frame}
