\documentclass[11pt]{article}

\usepackage{graphicx}
\usepackage{amsmath,amsthm,amssymb,color,latexsym}
\usepackage{enumitem,cancel}
\usepackage{listings}
\usepackage{xcolor}
\usepackage{hyperref}
\hypersetup{
	colorlinks=true,
	linkcolor=blue,
	filecolor=magenta,
	urlcolor=cyan,
	pdftitle={Biomolecular Structures},
	pdfpagemode=FullScreen,
}

\tolerance 99999
\hbadness 99999

\definecolor{codegreen}{rgb}{0,0.6,0}
\definecolor{codegray}{rgb}{0.5,0.5,0.5}
\definecolor{codepurple}{rgb}{0.58,0,0.82}
\definecolor{backcolour}{rgb}{0.99,0.99,0.97}

\lstdefinestyle{mystyle}{
	backgroundcolor=\color{backcolour},
	commentstyle=\color{codegreen},
	keywordstyle=\color{magenta},
	numberstyle=\tiny\color{codegray},
	stringstyle=\color{codepurple},
	basicstyle=\ttfamily\footnotesize,
	breakatwhitespace=false,
	breaklines=true,
	captionpos=b,
	keepspaces=true,
	numbers=left,
	numbersep=5pt,
	showspaces=false,
	showstringspaces=false,
	showtabs=false,
	tabsize=2
}

\lstset{style=mystyle}


% Margins
\topmargin=-0.45in
\evensidemargin=0in
\oddsidemargin=0in
\textwidth=6.5in
\textheight=9.0in
\headsep=0.25in

\newenvironment{points}{\vspace{0.3cm} \begin{enumerate}[label={(\alph*)}]}{\end{enumerate} \vspace{0.2cm}}
\newenvironment{solution}{\subsection*{Solution:}}{\vspace{0.5cm} \hrule \vspace{0.7cm}}
\newtheorem{problem}{Problem}


\title{ Summer Project 2024 \\ Report }
\author{ Abhiram Tilak - 2022113011 }
\date{\today}

\begin{document}
\maketitle

\section{Summary}

The initial description of my summer project was to read given chapters of two
books, \texttt{Introduction to Cosmology - Barbara Ryden } and the
\texttt{Introductionn to Elementary Particles - David Griffiths}. But when
reading certain chapters of Ryden, I noticed that I don't really have the
mathematical background for understanding certain topics like tensors, curvature and
special theory of relativity. So following my advisor's recommendation, I
had to read certain chapters of
\texttt{Mathematical Methods for Physicists - George B. Arfken}, and also some
parts of \texttt{Introduction to Electrodynamics - Griffiths} to make it easier
for me to understand concepts. After doing so
I have completed the required chapters in cosmology, but for the second book
in the reading project, I have only covered two chapters. Thus in this report
I am going to include both Cosmology and related Mathematics that I have
learned.

\section{Problems}

As instructed by my advisor, I have to pick a few problems and solve them
as a part of my report. I have picked these problems from a combination of books
as referenced.

\begin{problem}
	Suppose that $H$ scales as temperature squared all the way back until
	the time when the temperature of the universe was $10^{19} GeV/k_B$
	(universe was dominated by radiation all the way upto planktime). Also
	suppose that dark energy is in the form of a cosmological constant
	$\Lambda$, such that $\rho_\Lambda$ remains constant throughout the
	history of the universe. What was $\rho_\Lambda/(3H^2/8\pi G)$, back
	then? (Source: \texttt{Exercise 1.1: Modern Cosmology - Scott Dodelson} )
\end{problem}

\begin{solution}
	Given $H$ scales as temperature squared:
	$$ H(t) \propto T^2 $$
	$$ \frac{H(t)}{H_0} \propto \left(\frac{T}{T_0}\right)^2 $$
	Given $T = 10^{19} GeV/k_B$, converting into kelvin:
	$$ T = 10^{19} GeV/k_B = 10^{19} \times \frac{1.6 \times 10^{-10}}{1.38
	\times 10^{-23}} K = 1.159 \times 10^{32} K$$
	Using $T_0 = 2.7 K$ (CMB radiation), we have
	$$ \left(\frac{T}{T_0}\right) = \frac{1.159 \times 10^{32} }{2.7} = 4.29
	\times 10^{31}$$

	Thus the value, $\rho_\Lambda/(3H^2/8\pi G) $ can be given as:
	$$ \frac{\rho_\Lambda}{\left(\frac{3H^2}{8\pi G}\right)} =
	\frac{\rho_\Lambda}{\rho_{c,0}} \times \left( \frac{H_0}{H} \right)^2$$

	Using the above relation between H and T:
	$$ \frac{\rho_\Lambda}{\left(\frac{3H^2}{8\pi G}\right)} =
	\frac{\rho_\Lambda}{\rho_{c,0}} \times \left( \frac{T_0}{T} \right)^4$$

	Writing $\frac{\rho_\Lambda}{\rho_{c,0}}$ as $ \Omega_{\Lambda, 0} =
	0.7$, (cosmological cosmological constant remains fixed).

	$$ \frac{\rho_\Lambda}{\left(\frac{3H^2}{8\pi G}\right)} =
	0.7 \times \left( \frac{1}{4.29 \times 10^{31}} \right)^4$$

	$$ \frac{\rho_\Lambda}{\left(\frac{3H^2}{8\pi G}\right)} =
	9 \times 10^{-128}$$
\end{solution}


\begin{problem}
	Convert the specific intensity in Eq. (1.9) into an expression for what
	is plotted in
	Fig. 1.7, the energy per area, time, frequency and steradians. Show that
	the peak of a
	2.73 K black-body spectrum does lie at $1/\lambda \simeq  5 cm^{-1}$. What frequency does this
	correspond to? (Source: \texttt{Exercise 1.4: Modern Cosmology - Scott Dodelson} )


	For reference Eq. (1.9) is given by:
	$$ I_\nu = \frac{4\pi\hbar}{c^2} \frac{\nu^3}{e^{2\pi\hbar\nu/k_BT} - 1} $$

	And Fig. 1.7 is:
	\center
	\includegraphics[width=4in]{1}
\end{problem}

\begin{solution}
	By differentiating equation by $\nu$ we get
	$$ \frac{d I_\nu}{d\nu} = \frac{4\pi\hbar}{c^2} \frac{d}{d\nu}\left(
	\frac{\nu^3}{e^{2\pi\hbar\nu/k_BT} - 1}\right) $$
	$$ \frac{d I_\nu}{d\nu} =\frac{4\pi\hbar}{c^2} \left(
		\frac{3\nu^2 (e^{2\pi\hbar\nu/k_BT} - 1) - \nu^3
		(e^{2\pi\hbar\nu/k_BT} - 1) (\frac{2\pi\hbar}{k_B T})}{ (e^{2\pi\hbar\nu/k_BT} - 1)^2}
	\right)$$

	$$ \frac{d I_\nu}{d\nu} =\frac{4\pi\hbar}{c^2} \left(
		\frac{3\nu^2  - \nu^3
		(\frac{2\pi\hbar}{k_B T})}{ e^{2\pi\hbar\nu/k_BT} - 1}
	\right)$$


	$$ \frac{d I_\nu}{d\nu} =\frac{4\pi\hbar \nu^2}{k_B Tc^2} \left(
		\frac{3k_B T  - \nu
		(2\pi\hbar)}{ e^{2\pi\hbar\nu/k_BT} - 1}
	\right)$$

	Thus when equating it to zero to find the peak
	$$ \frac{d I_\nu}{d\nu} = 0 = 3k_B T  - \nu
	(2\pi\hbar)$$
	This gives $$ v_{max} = \frac{3k_B T}{2\pi \hbar} $$

	When $T = 2.73 K$, we get
	$$ v_{max} = \frac{3(1.38 \times 10^{-23}) (2.73)}{6.6 \times 10^{-34}} $$
	$$ v_{max} = 1.68 \times 10^{11} Hz $$

	Using $ \lambda = \frac{c}{\nu}$
	$$ \frac{1}{\lambda_{max}} = \frac{1.68 \times 10^{11}}{3\times 10^8}
	m^{-1} $$
	$$ \frac{1}{\lambda_{max}} = 5.59 \times 10^{2}m^{-1} $$
	$$ \frac{1}{\lambda_{max}} \approx 5 cm^{-1} $$

	Now in the given diagram the units used is $MJy/sr$ which is Mega Jansky
	per steradians, whose dimension is similar to RHS

	$$ 1 Jy = 10^{-26} SI units $$
	Thus
	$$ Intensity[MJy/sr] = 10^{20} I_\nu (SI units) $$
	Which is the same units used in the graph.


\end{solution}

\begin{problem}
	At early times, the cosmological constant can be neglected. Using this
	approximation, integrate Eq. (1.3) in a Euclidean universe to obtain a(t). Using
	$T (t) = T_0/a(t)$,
	determine the times when the cosmic temperature was 0.1 MeV and 1/4 eV.
	(Source: \texttt{Exercise 2.5: Modern Cosmology - Scott Dodelson} )

	For reference: Eq. (1.3) is the Friedmann equation:
	$$ H^2(t) = \frac{8\pi G}{3} \left[ \rho(t) + \frac{\rho_{cr} -
	\rho(t_0)}{a(t)^2} \right] $$
\end{problem}

\begin{solution}

	We know that the Hubble rate is defined as:
	$$ H(t) = \frac{\dot a }{a} = \frac{1}{a} \frac{da}{dt} $$

	Using this time can be written as:
	$$ dt = \frac{1}{a} \frac{da}{H(t)} $$
	When value of $a$ goes from 0 to a.
	$$ t = \int^a_0 \frac{1}{a} \frac{da}{H(t)} $$

	Since the we ignore curvature and the cosmological constant we have:
	$$ \left(\frac{H(t)}{H_0}\right)^2 = \frac{\Omega_{m,0}}{a^3} + \frac{\Omega_{r,0}}{a^4}  $$

	Using this in the above equation we get:
	$$ t = \frac{1}{H_0} \int^a_0 \frac{1}{a} \frac{da}{\sqrt{\frac{\Omega_{m,0}}{a^3} + \frac{\Omega_{r,0}}{a^4}}} $$
	$$ H_0 t = \frac{1}{\sqrt{\Omega_{r,0}}} \int^a_0
	\frac{a.da}{\sqrt{\frac{\Omega_{m,0}a}{\Omega_{r,0}} + 1}} $$

	We can define $a_{rm} = \frac{\Omega_{r,0}}{\Omega_{m,0}}$
	$$ H_0 t = \frac{1}{\sqrt{\Omega_{r,0}}} \int^a_0
	\frac{a.da}{\sqrt{\frac{a}{a_{rm}} + 1}} $$
	We can use Taylor expansion to integrate this, and ignore higher terms:
	$$ H_0 t = \frac{4a^2_{rm}}{3\sqrt{\Omega_{r,0}}}\left[ 1 - \left(1 -
	\frac{a}{2a_{rm}}\right)\left( 1 + \frac{a}{a_{rm}} \right)^{1/2} \right] $$

	Since the given temperatures we can estimate $a << a_{rm}$, when
	radiation dominates.
	$$ t  \approx  \frac{a^2}{2H_0\sqrt{\Omega_{r,0}}}$$

	Using the temperature equation given in the question:
	$$ T(t) = \frac{T_0}{a(t)} $$
	$$ a(t) = \frac{T_0}{T(t)} $$
	$$ a(t) = \frac{k_B T_0}{k_B T(t)} $$
	Taking $k_B T_0 = 2.35\times 10^{-4} eV$ (CMB), and using

	(i) $T(t) = 0.1 MeV = 10^5 eV$
	$$ a(t) = 2.35 \times 10^{-9} $$
	$$ t  =  \frac{a^2}{2H_0\sqrt{\Omega_{r,0}}}$$
	Now taking $\Omega_{r,0} = 9 \times 10^{-5}$, $H_0^{-1} = 14.4 Gyr$
	$$ t  = 4.19 \times 10^{-6} yr $$
	$$ t  = 132 s $$

	(ii) $T(t) = 0.25 eV$
	$$ a(t) = 9.4 \times 10^{-4} $$
	$$ t  =  \frac{a^2}{2H_0\sqrt{\Omega_{r,0}}}$$
	$$ t  = 3.88 \times 10^5 yr $$


\end{solution}

\begin{problem}
	How is the energy density of a gas of photons with a black-body spectrum
	related to
	the specific intensity of the radiation? That is, what is the relation
	between $\rho_{\gamma}$ and $I_{\nu}$
	defined in Eq. (1.9) (Source: \texttt{Exercise 2.8: Modern Cosmology - Scott Dodelson} )

	For reference Eq. (1.9) is given by:
	$$ I_\nu = \frac{4\pi\hbar}{c^2} \frac{\nu^3}{e^{2\pi\hbar\nu/k_BT} - 1} $$


\end{problem}

\begin{solution}
	WKT Energy density from frequency $\nu$ to $\nu+d\nu$, is given by:
	$$ \epsilon(\nu)d\nu = \frac{8\pi h}{c^3}\frac{\nu^3d\nu}{e^{h\nu/k_BT} - 1} $$
	$$ \epsilon(\nu) d\nu= \frac{16\pi^2\hbar}{c^3}\frac{\nu^3d\nu}{e^{2\pi
	\hbar \nu/k_BT} - 1} $$
	$$ \epsilon(\nu) d\nu= \frac{4\pi}{c} I_\nu d\nu$$

	Writing $E_{mean} = pc = h\nu$ for radiation, we have
	$$\frac{2\pi}{c} d\nu = dp \frac{1}{\hbar} $$
	$$ \epsilon(\nu) d\nu= \frac{2}{\hbar} I_\nu dp$$

	By integrating over all frequencies we have
	$$ \epsilon = \frac{2}{\hbar} \int_{0}^{\infty} I_\nu dp$$

\end{solution}

\begin{problem}
	A hypothesis once used to explain the Hubble relation is the “tired
	light
	hypothesis.” The tired light hypothesis states that the universe is not
	expanding, but that photons simply lose energy as they move through
	space (by some unexplained means), with the energy loss per unit
	distance being given by the law

	where k is a constant.
	$$ \frac{dE}{dr} = -kE $$
	Show that this hypothesis gives a distance–
	redshift relation that is linear in the limit $z <<1$. What must the value
	of
	k be in order to yield a Hubble constant of $H_0 = 68 km s^{-1} Mpc^{-1}$?
	(Source: \texttt{Exercise 2.4: Introduction to Cosmology - Barbara Ryden} )

\end{problem}


\begin{solution}
	Given equation,
	$$ \frac{dE}{dr} = -kE $$
	can be written as
	$$ E(r) = E_0 e^{-kr} $$

	For light from stars we can write energy in terms of wave length as
	$\lambda  = \frac{hc}{E}$

	We can write redshift in terms of this wavelength as:
	$$ z = \frac{\lambda_e - \lambda_0}{ \lambda_0 } = \frac{\frac{hc}{E} -
	\frac{hc}{E_0}}{ \frac{hc}{E_0}}$$

	$$ z = e^{kr} - 1 $$
	Comparing this with the Hubble's equation:
	$$ z = \frac{H_0}{c} r  $$
	We have:
	$$ \frac{H_0}{c}r = e^{kr} - 1 $$
	By using taylor expansion:
	$$ \frac{H_0}{c}r \approx kr $$
	$$ k = \frac{H_0}{c} = 2.3 \times 10^{-4} Mpc^{-1}$$

\end{solution}

\begin{problem}
	The principle of wave-particle duality tells us that a particle with momentum
	p has an associated de Broglie wavelength of $\lambda = h/p$; this
	wavelength increases as $\lambda \propto a$ as the universe expands. The total
	energy density of a gas of particles can be written as $\epsilon = nE$, where n is
	the number density of particles, and E is the energy per particle. For
	simplicity, let’s assume that all the gas particles have the same mass
	m and momentum p. The energy per particle is then simply
	$$ E = (m^2c^4 + p^2 c^2)^{1/2} =   (m^2c^4 + h^2 c^2 / \lambda^2)^{1/2}$$
	Compute the equation-of-state parameter w for this gas as a function
	of the scale factor a. Show that $w = 1/3$ in the highly relativistic limi
	$(a \rightarrow 0, p \rightarrow \infty)$ and that $w = 0$ in the highly nonrelativistic limit
	$(a \rightarrow \infty, p \rightarrow 0)$.
	(Source: \texttt{Exercise 4.5: Introduction to Cosmology - Barbara Ryden} )

\end{problem}

\begin{solution}
	We can write Energy in terms of energy density as:
	$$ \epsilon V = (m^2c^4 + h^2 c^2 / \lambda^2)^{1/2} $$
	$$ \epsilon  = \frac{1}{V} (m^2c^4 + h^2 c^2 / \lambda^2)^{1/2} $$
	We know that volume scales cubically and, $\lambda$ scales linearly:
	$$ \epsilon  = \frac{1}{Va^3(t)} \left(m^2c^4 + \frac{h^2 c^2}{a^2(t)
	\lambda^2} \right)^{1/2} $$

	Taking log both sides
	$$ \epsilon  =\frac{1}{V_0 a^3(t)} \left(m^2c^4 + \frac{p^2 c^2}{a^2(t)} \right)^{1/2} $$
	$$ ln(\epsilon)  = -ln V_0 - 3 ln(a(t)) + \frac{1}{2} ln\left(m^2c^4 + \frac{p^2 c^2}{a^2(t)} \right) $$
	Differentiate this equation by time:
	$$ \frac{\dot \epsilon}{\epsilon}  = - 3 \frac{\dot a}{a} + \frac{1}{2}
	\frac{ -2 \frac{p^2c^2 }{a(t)^3} \dot a}{\left(m^2c^4 + \frac{p^2 c^2}{a^2(t)} \right)} $$
	$$ \frac{\dot \epsilon}{\epsilon} + 3 \frac{\dot a}{a} = - \frac{ p^2
	}{\left( a(t)^2 m^2c^2 + p^2 \right)}\frac{\dot a}{a} $$
	$$ \frac{\frac{\dot \epsilon}{\epsilon} + 3 \frac{\dot a}{a}}{\frac{3\dot a}{a}} = - \frac{ p^2
	}{3\left( a(t)^2 m^2c^2 + p^2 \right)} $$

	Now using fluid equation and equation of state:
	$$ \dot \epsilon + 3 \frac{\dot a}{a}(\epsilon + P) = 0$$
	$$ \dot \epsilon =- 3 \frac{\dot a}{a}(1+w)\epsilon$$
	$$ \frac{\dot \epsilon}{\epsilon} =- 3 \frac{\dot a}{a}(1+w)$$
	$$ \frac{\frac{\dot \epsilon}{\epsilon}}{\frac{3\dot a}{a}} =- (1+w)$$
	$$ - \left(\frac{\frac{\dot \epsilon}{\epsilon}  +
		\frac{3\dot a}{a}}{\frac{3\dot a}{a}}
	\right) =w$$

	Thus
	$$ w = \frac{p^2}{3\left( a(t)^2 m^2c^2 + p^2 \right)}  $$
	When $p\rightarrow \infty$
	$$ w = \frac{1}{3\left( a(t)^2 \frac{m^2c^2}{p^2} + 1 \right)}  $$
	$$ w = \frac{1}{3}  $$
	When $p\rightarrow 0$
	$$ w = 0 $$
\end{solution}

\begin{problem}

	The components of tensor A are equal to the corresponding components of
	tensor B in
	one particular coordinate system denoted, by the superscript 0; that is,
	$$ A_{ij}^0 = B_{ij}^0 $$
	Show that tensor A is equal to tensor B, $A_{ij} = B_{ij}$ , in all coordinate
	systems
	(Source: \texttt{Exercise 4.1.2: Mathematical Methods for Physicists -
	George B. Arfken} )
\end{problem}

\begin{solution}
	Let us define a coordinate transformation from the coordinate system
	$x^0_{ij}$ to say $x_{\alpha \beta}$.

	Thus for any given vector $A^0_{ij}$ can be written in terms of the new
	coordinate system as:
	$$ A_{\alpha \beta} = \frac{dx_{\alpha}}{dx^0_{i}}
	\frac{dx_{\beta}}{dx^0_{j}} A^0_{ij} $$

	Since, $A^0_{ij} = B^0_{ij}$
	$$ A_{\alpha \beta} = \frac{dx_{\alpha}}{dx^0_{i}}
	\frac{dx_{\beta}}{dx^0_{j}} A^0_{ij} $$

	$$ A_{\alpha \beta} = \frac{dx_{\alpha}}{dx^0_{i}}
	\frac{dx_{\beta}}{dx^0_{j}} B^0_{ij} $$

	Since the same transformation equation also applies to $B^0_{ij}$
	($A^0 and B^0$) are defined in the same coordinate system:

	$$ A_{\alpha \beta} = B_{\alpha \beta} $$

	Therefore if components of tensor A and B are equal under one coordinate
	system then they are also equal under any coordinate system as long as
	we can define a transformation property between those coordinate systems.
\end{solution}

\section{Book Summary}

I have read parts of 3 different books as a part of my summer project and I
would like to include a brief summary of all the chapters I've read knowledge
that I've gained.

\subsection{Introduction to Cosmology}

This book is aim at undergrad students, and is fairly recent in
terms of its contents. Unlike other cosmology books, this book focusses on
building of intuition stepwise, and is far less math intensive at
the same time avoids over-simplification of concepts.

\paragraph{Chapter 1: Introduction}

The book begins by laying a solid foundation for the study of cosmology:

\begin{itemize}
	\item Introduces various notations and units commonly used in the field
	\item Covers different units for mass, time, distance, and luminosity
	\item Briefly touches on Planck units, though they are not extensively used later
	\item Provides a philosophical and historical context for cosmology, tracing its roots
		to the pre-modern era
\end{itemize}

\paragraph{Chapter 2: Fundamental Observations}

This chapter delves into the core astronomical observations that form the basis of modern cosmology:

\begin{itemize}
	\item Presents key astronomical observations and their current explanations
	\item Discusses how these observations lead to various postulates and
		theories and introduces different observational methods used by
		astronomers and cosmologists
	\item Some important concepts discussed include:
		\begin{itemize}
			\item Olber's paradox: Which stems from an observation
				that 'sky is dark' and then builds up on
				different conclusion that can be made about
				current state of universe
			\item Homogeneity and Isotropy: An important concept
				which allows us to make very useful
				approximations about the state of current
				cosmos.
			\item Hubble's Law: Starting from redshift this law is
				sets the record for most of the mathematical
				calculations in cosmology. Also introduces
				different models of universe like Big Bang and
				Steady-state model, goes over different aspects
				of each.
			\item Cosmic Wave Background: Is important observation
				which leads us into using the Big-Bang model for
				rest of the book.
		\end{itemize}
	\item Covers essential prerequisites, including Properties of different particles and Characteristics of neutrinos and photons
	\item Sets the theoretical groundwork for the rest of the book

\end{itemize}

\paragraph{Chapter 3: Newton Vs. Einstein}

Since cosmology involves using a mixture of classical Newtonian mechanics but at
the same time taking advantage of modern concepts like relativity, a chapter
like this is mandatory to provide a brief overview of both classical and
derived laws of physics useful for our calculations.

\begin{itemize}
	\item Chapter starts by going over historical evolution of the concept
		of gravity and how Newton interpreted gravity along with
		Poisson's equation.
	\item Dedicates a whole section to touch upon newton's special theory of
		relativity, by taking different examples and introducing
		space-time and Minkowski metrics. Does a good job in slowly
		introducing relativity without nose-diving into complexities
		like four vectors and Lorentz transformations
	\item Builds up intuition for space-time curvature and General Theory of
		Relativity using various examples involving Elevators and light
		beams.
	\item Introduces curvature, but doesn't involve proofs for most of the
		results but just touches on fundamental results for measuring
		length in uniform positive and negative curved spaces.
	\item Finally shows the General Theory equivalent of Minkowski metric
		(Robertson-Walker metric) and how we can use that to find
		something called "proper distance", which is the standard way
		that we will be measuring cosmological distances moving forward.
\end{itemize}

\paragraph{Chapter 4: Cosmic Dynamics}

This is the point where it feels like the core Cosmological concepts and
required equations and laws are introduced. Introduces various notations
used throughout Cosmology and also involves proves for various laws and theorems
that are mentioned in text.

\begin{itemize}
	\item Starts off by dropping the Einstein's Field Equation, and goes on about how it
is difficult to solve it and there needs to be a better estimate which relates
the energy density with terms like curvature and scaling factor.
\item Then there is a
step by step proof building up on the so called "Friedmann Equation" right from
the Newton's Gravitational law, which sets
the premise for all analysis that we perform on the present universe.
\item Similarly there is another fairly important equation called "Fluid Equation" or
the "Continuity Equation", which can be derived from the laws of Thermodynamics.
Using these both equations, we derive the "acceleration equation" or the "Second
Friedmann Equation".
\item Finally we come up with a relation called "state" equation, which in itself is
not that useful, except we can solve the equation for different components.
\item After introducing all the above equations, the book takes us to analysis
different components of universe and comes up with a controversial "dark energy"
term that equalizes the failed measurements of Hubble's constant.
\end{itemize}

\paragraph{Chapter 5: Model Universes}

Applying the knowledge from the previous chapter about different components
that make up universe like (matter, radiation, dark energy), we can now create
"model" universes, which are special cases of our present universe. We can
derive useful results which can help us arrive at interesting observations and
theories which are opposed to the currently proposed Big Bang model.

\begin{itemize}
	\item Firstly, the chapter solves the state equation and the fluid equation to arrive at a
		relation between energy density of a component vs. its
		dependence on scaling factor of the universe, which is useful to
		analyse how different components scale
	\item Use the above relation to solve the Friedmann equation to arrive
		at a new form of the equation which is particularly useful for
		analysing the rate of universe expansion. We also write the
		curvature term in terms of critical density since it gives a
		better split between ratios of different components.
	\item Model universes are of different types; Empty Universe (only
		curvature), Single Component (flat universe, only one component)
		and Multi-Universe system which are mixture of any two or three
		of above. The main reason we adopt such a differential splitting
		is because solving the equation with 4-5 terms and taking
		integral can be difficult.
	\item A common method to solve these model universes is to find a
		relation between scaling factor and time and then using it to
		calculate the age of the universe and find good estimations for
		proper distance.
	\item Using the conclusions drawn from above we can create a "Benchmark"
		universe which is a rough estimate of the present universe which
		is obtained by results from above  model universes.

\end{itemize}

\paragraph{Chapter 6: Solving for Cosmological Parameters}

This chapter goes over how most of the conclusions about the universe can be
drawn by solving for just two constants in the universe $H_0$ (Hubble's constant)
and $q_0$ (deceleration Constant). We already know of the Hubble's constant but
the deceleration constant is basically the second degree coefficient obtained
after Taylor expanding the expansion coefficient. There are many ways in which
astronomers can make observations and attempt to estimate these both values.

\begin{itemize}
	\item When we solve for proper distance in terms of $q_0$ and $H_0$, we
		get it in terms of lookback time, but the book mentions how this
		is not a good metric and this needs to be converted to a more
		measurable quantity like redshift.
	\item Then the chapter shows some popular methods in which cosmological
		distances are measured by astronomers like "Angular Diameter
		Distance", "Luminosity Distance" and "Bolometric Distance".
		All these distances measured have to be scaled when scaling of
		the universe it taking into account.
	\item All above forms of distance measures have their own advantages and
		disadvantages but Luminosity distance is the most widely used,
		ones and require us to have a "standard candle", whose
		luminosity is known to us, to help measure the value.
	\item The instrument used to measure this distance is called Bolometer.
		There are many celestial objects that can serve as Standard
		candles but Cepheids (10 Mpc range) and Type IA Super Nove stars
		(> 100 Mpc ), act as good candles, because of their high and periodic
		luminosity properties.
\end{itemize}

\paragraph{Chapter 7: Dark matter}

I didn't really get to have a full presentation of this chapter, but I still
read it, so here's a summary.

This chapter focusses on constructing the idea of dark matter, what it is and
how it is represented mathematically. The chapter first tries to perform
estimations of Stellar Density (energy density corresponding to stars), using
the luminosity to mass ratio, and other known quantities like initial mass
index. Then the it shows some popular astronomical observations made on a far
cluster called "COMA" clusters. X-ray spectroscopy reveals that stellar mass of
the galaxy just accommodates upto 5\% of the matter density. A lot of it is just
cosmic gas floating around between clusters and galaxies, but even after
considering these cosmic waste and gases we only get upto fifth of the matter
density. The book then finally
defines dark matter, as matter which do not interact with
like electromagnetic fields and hence difficult to detect.


\subsection{Mathematical Methods of Physicists - Tensors}

I was told to read up some parts of Tensors from this book until I am familiar
enough with the covariant and contravariant notations and
Christoffel Symbols, etc.


\paragraph{Covariant and Contravariant Tensors:}

Covariant tensors transform with the coordinate system, while contravariant tensors transform inversely.
Covariant tensor components are denoted with subscripts, contravariant with superscripts.
The distinction is crucial in general relativity and differential geometry.

\paragraph{Basic Operations on Tensors:}

Tensors of the same rank and type can be added or subtracted component-wise.
The result is a tensor of the same rank and type as the operands.
This operation is only defined for tensors in the same coordinate system.

\paragraph{Symmetry and Isotropic Tensors:}

A tensor is symmetric if it remains unchanged when any pair of its indices are interchanged.
Symmetry reduces the number of independent components in a tensor.
Many physical quantities, like the stress tensor, are represented by symmetric tensors.

Isotropic tensors have components that are invariant under rotations of the coordinate system.
They are used to describe properties of materials that are the same in all directions.
The Kronecker delta and Levi-Civita symbol are examples of isotropic tensors.

\paragraph{Contraction and Direct Product:}

Contraction reduces a tensor's rank by summing over a pair of indices.
It's often used to extract scalar quantities from higher-rank tensors.
The trace of a matrix is an example of tensor contraction.

The direct product combines two tensors to form a higher-rank tensor.
It's denoted by the symbol $\otimes$ and is also called the tensor product.
The resulting tensor's rank is the sum of the ranks of the input tensors.

\paragraph{Inverse Transformations:}

Inverse transformations convert contravariant tensors to covariant tensors and vice versa.
They are crucial for maintaining tensor equations' form invariance under coordinate changes.
The metric tensor and its inverse play a key role in these transformations.

\section{Conclusion}

The nature of my research field requires a strong foundation in advanced
Mathematics and Physics. While many Dual-Degree students have begun engaging
with current literature, my initial focus will be on acquiring the requisite
theoretical background and reading various text-books. Consequently, the majority of my first-year research
activities will involve studying theory. I feel this preparatory phase is essential before transitioning into an
exploration of state-of-the-art research in the field.

\end{document}
